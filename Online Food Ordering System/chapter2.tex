\begin{center}
   \section*{\fontsize{20}{20}\selectfont Chapter 2}
   \end{center}
\vspace{10mm}
\section{Objective}

The primary objective of an online food ordering system is to provide a convenient and efficient way for customers to order food from restaurants. Here are some specific objectives of an online food ordering system:
\begin{itemize}
\item \textbf{ Convenience:} Online food ordering systems make it easier for customers to order food without having to leave their homes or offices. Customers can place orders at any time, and the food will be delivered to their location.

\item \textbf{Accessibility:} Online food ordering systems make it easier for customers to access a variety of restaurants and menus in one place. Customers can browse different options and choose the one that best suits their needs.

\item \textbf{Speed:} Online food ordering systems can help to reduce the time it takes to place an order and receive food. Customers can place their orders quickly and easily, and the food will be delivered to their location promptly.

\item \textbf{Accuracy: }Online food ordering systems can help to reduce errors in ordering, such as miscommunications between the customer and the restaurant. Customers can place their orders online, and the system can relay the order accurately to the restaurant.

\item \textbf{Cost-effectiveness:} Online food ordering systems can help to reduce costs for both customers and restaurants. Restaurants can reduce their staffing costs, and customers can save money by avoiding travel expenses.
\end{itemize}
Overall, the objective of an online food ordering system is to provide a convenient, accessible, and efficient way for customers to order food from restaurants while reducing costs and increasing accuracy.

\subsection{Project Review}

The Online food odering system is a well-designed and comprehensive software application that provides businesses with an efficient and user-friendly solution to manage their online shops. The system offers a range of features, including inventory management, order processing, customer management, food odering and delivery management, to streamline the management of an online . One of the key strengths of the Online food odering system is its user-friendly interface,
making it easy for businesses to manage their online website, even without technical expertise. The system also offers real-time reporting, which allows businesses to make informed decisions about their online shop's performance. Another strength of the online food odering system is its scalability, which allows businesses to adapt and grow over time. The system can accommodate a range of business sizes and can handle a large number of products and customers. However, one potential weakness of these system is its cost, which may be prohibitive
for small businesses with limited budgets. Additionally, the system's reliance on an internet connection may be a concern for businesses . Overall, the Online food odering system is an excellent solution for businesses looking to optimize their online shop's performance. It offers a range of features and benefits that can help businesses streamline their operations, reduce costs, and improve customer satisfaction.
\\
\\
\textbf{Admin:} Admin can manage customer order which contains the order number, customer name, branch name, order date & time of all the branch, admin can maintain status report too whether the order delivered or accepted or rejected or still in pending There are some of the administrative tasks that can be performed using the Online food ordering system:

\begin{itemize}
    \textbf{1. Product Management: }The system allows the administrator to
    manage the products in the online shop, including adding new
    products , updating product information, and removing
    products from the inventory.\\
    \textbf{2. Order Processing: }The administrator can manage and process
    orders placed by customers using the system, including order
    verification, payment processing, and shipping and delivery
    management.\\
    \textbf{3. Customer Management:} The system allows the administrator
    to manage customer data, including purchase history, contact
    information, and feedback, to improve customer relationships
    and satisfaction.\\
    \textbf{4. Inventory Management:} The administrator can monitor
    inventory levels and receive real-time alerts for low stock levels,
    ensuring that products are available when customers place an
    order.\\
    \textbf{5. Reporting:} The system provides customizable real-time
    reports, allowing the administrator to track sales, revenue, and
    other metrics to make informed decisions about the online
    shop's performance.\\
    
    \textbf{6. User Management:} The administrator can manage user access to the system, setting user roles and permissions to  ensure the security of the online shop's data.\\
    \textbf{7. Marketing Management: }The administrator can manage
    marketing campaigns, including email marketing, social media
    marketing, and advertising campaigns to promote the online
    shop and increase sales.
    Overall, the online food ordering system provides a comprehensive
    solution for administrators to manage all aspects of an online shop.
 \end{itemize}


\textbf{User:}
Here is an example of a user workflow for the online food ordering system
 \begin{itemize}
    \textbf{1. Browse the Products:} Users can browse the products on the website using the search bar, filters or category list.\\
    
    \textbf{2. View Product Details:} Users can click on a product to see more details about it, such as the product description, price, images, reviews, and specifications.
    \\
    \textbf{3. Add to Cart:} Users can add the desired products to the shopping cart and continue shopping.
    \\
    \textbf{4. Checkout:} Users can review their cart, update the quantity, and proceed to checkout. At checkout, users can enter their shipping and billing information, and select the payment method.
    \\
    \textbf{5. Order Confirmation:} Once the payment is made, the user will receive an order confirmation via email and will be redirected to the order confirmation page on the website.

 \end{itemize}
